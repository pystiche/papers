\section{Methodology of Replicability}
In this study, we have implemented the \gls{NST} approaches presented in \secref{sec:replicated_paper} based on the \gls{NST} library \texttt{pystiche} \cite{ML2020} in \texttt{pyTorch}. Although the authors have incorporated further images from additional experiments in the paper, we have only replicated the images that show the result of the underlying approach in order to determine whether the replication of the approach is successful. 

As mentioned earlier, we found differences between the approach published in the paper and the reference implementation in all replications. In most replications, these differences have an impact on the result, either without stylisation at all, the result looks visually different, or only unusable images are created. For this reason, the next section explains which differences were found. It also explains how we dealt with these differences. Then the information on the images and datasets used for each approach is given. Finally, we discuss the computational requirements we used for this study. 

\subsection{Parameter deviations} \label{sec:replicability}
We replicated each approach once with the information from the paper and once with the default values from the reference implementation. An overview of the parameters used can be found in the tables in the \appref{sec:parameters_appendix}. The tables contain the information of the authors published in their paper and the default values from the implementation. Furthermore, a detailed documentation of the hyperparameters used can be found in the documentation of \texttt{pystiche/papers}\footnoteref{docs}. 

The deviations between the parameters described in the paper and the original implementation are marked by comments. In addition, links are provided with the location of the deviation in an archived version of the original implementation at the time of this replication study. The following designation is chosen to distinguish between the original paper and the reference implementation: 
\begin{itemize}
	\item \paper{} -- Refers to the replication only with the information from the original paper. 
	\item \implementation{} -- Refers to the replication only with the information from the reference implementation. 
\end{itemize}

In general, the deviations can be divided into two categories. The first category is the behavioral changes that probably result from author misconceptions about how their used framework or library works internally. The second category \eqq{hyperparameters} result from the different specifications in the documentation and the finally implemented approach. 

Another influencing factor for this category is the high number of different hyperparameters needed for the implementation. These range from the I/O periphery of the images, to the pre- and postprocessing methods applied, to the specific hyperparameters for the loss functions, which makes it complex to document accurately. Information on both types of changes can be found for each paper implementation in the respective \eqq{Behavior Changes} and \eqq{Hyperparameters} sections of the documentation. 

In the following, we describe these categories and subcategories in more detail and explain how we deal with them. The examples do not appear in all replications, but serve only to illustrate the observed deviations in this replication study.

\subsubsection{Behavioral changes}
The behavioral changes are modified parameters that result from different specifications in \paper{} and \implementation{}. However, there should be no influence on these parameters to change the actual style transfer.

This type of error includes for example the use of the \gls{MSE} in \implementation{}, while the \gls{SE} is specified in \paper{}. This category also includes additional norms, such as a division of the number of channels, as well as additional pre-factors. The latter is mainly due to the specification of additional factors in the paper in the loss function which makes the expression of the gradient prettier, but are not integrated by default in the loss functions of the \gls{DLF} used in \implementation{}. 

An example of this is a factor $1/2$, which cancels out the $2$ from the square term in the gradient of the \gls{MSE} loss function. All these examples have an influence on the calculation of the loss functions. This changes the weighting between the content and the style, which can result in differences, as shown in \figref{fig:weighting_nst}. 

Another example of the behavioral changes are changes to the transformer structure in the model-based approaches. This can be different indications of the value range delimiter for the transformer output or different activation functions (LeakyReLU instead of ReLU). There may also be differences in the basic structure of the transformer by using additional or fewer blocks than specified in \paper{}. These differences influence what kind of transformation is learned in the transformer. This changes thus have an influence on the behaviour of the algorithm, but are not intended to have a direct influence on the style transfer.

All these behavioural changes are binary, whether from \paper{} or \implementation{}. We have therefore implemented both variants for the respective replication, and it is possible to switch between them.

\subsubsection{Hyperparameters}
In contrast to the behavioral changes, the changes of the category Hyperparameter are freely selectable and are suitable for influencing the style transfer positively by adapting. These parameters include the learning rate, the used layers from the encoder, the used loss weights of the individual loss functions, the number of optimisation steps, as well as other approach-specific parameters. 

Unfortunately, there are different information about the parameters used in \paper{} and \implementation{}. The differences range from different values, different information about the used interpolation or layers, to no specific information about the value at all. However, these parameters are essential for accurate replication, as shown in \secref{sec:preliminaries}. 

One reason for the differences is that it is difficult to properly document all these parameters and know exactly which parameters were used for a particular image/model, as the code is constantly changing before and sometimes after release. It can also become complex to correctly document all standardisations and mean calculations in the formulas in \paper{}, which may result in additional or missing factors. 

An example of this category are the layers used from the pre-trained networks. There are different statements about the depth of the layers, the number of the layers used and the feature map, i.e. whether the features before or after the activation function are selected. The fact that the choice of layers has an influence on the result has already been shown in \figref{fig:diff_depth_nst}. 

Another observed point of this category of changes are different information about important settings, such as the choice of starting point and the number of iteration steps. These are essential parameters for accurate replication, as shown in \figref{fig:iteration_nst} and \figref{fig:diff_init_nst}. However, it could be that such an important parameter is not specified in \paper{} and can only be taken from \implementation{} or two fundamentally different information are given.  

All these parameters have an influence on the final result and are necessary to determine if the replication is correct and if differences are due to rounding errors or different initialisations. For this reason, similar to the behavioral changes, we have used the information given in \paper{} and \implementation{} as default values. An overview of the parameters used for the respective replications can be found in the tables in \appref{sec:parameters_appendix}. In the case of missing information, we have used the available information for both variants. 

Since these parameters are intended to improve the stylisation, it is also possible that different parameters have been used for a given style image. These variations from the default values are indicated for the respective images or model trainings.

\subsubsection{Pre-trained Encoder}
Another important point for the replication of \gls{NST} approaches is the choice of the pre-trained encoder $E$. There are different architectures that can be used for the calculation of the loss functions. The most commonly used models are those of the \gls{VGG} \cite{SZ2015}. Some \gls{DLF} provide their own pre-trained weights for these networks. It is therefore important to choose the same settings as in the original approach. Otherwise, the features used for the optimisation for the images do not match and the optimisation process can no longer match.

This also applies to the necessary pre- and postprocessing steps for these pre-trained encoders. This data preparation involves a standardisation with the network-specific mean and standard deviation. However, the replicated paper rarely provide information on whether and when this step will be taken. This makes it difficult to trace this step, especially with model-based approaches, so that different implementations can occur.  

We have used and initialised the encoders according to the cited sources in the original papers. The Pre- and postprocessing are steps that are implied by the pre-trained encoders used, so we have integrated this step at the places where the original authors have done this step.

\subsubsection{Transformation of the images}
The images in the datasets or the images used in the image-based methods often do not exist in the size used for the approach. The images are therefore transformed to the size used. Sometimes the transformation is also necessary or useful to make the approach work at all or to provide usable images, such as images with a square image size of $2^{m}$ with $ m \in \mathbb{Z}^{+}$, so that the transformer does not change the image size due to discretisation. 

However, in \paper{} there is often little or no information about these steps available. But in \implementation{}, the images are resized to the final image size by several transformations, such as resizing and cropping. This also applies to training time. The training images used are modified during the training by numerous image augmentation. This means that the used dataset is not only resized to the right size, but also artificially enlarged by random transformations, such as rotation, translation and scaling, but also random changes in the light level or brightness of the images. Especially the latter may have an influence on the way the style is presented, as it is available in different sizes and orientations for the training. The effects of different image sizes have already been shown in \figref{fig:diff_size_nst}. As a result, it is not possible to trace exactly how the used image was created. In addition, the training in the model-based methods is altered by the artificial enlargement of the dataset, which may also have an influence on the result.

For this replication study, we have re-implemented these transformations and use them for \implementation{}. However, since this is a comprehensible step even without information, we use a minimally necessary process, in the case that no further information is given. This means that if no further information is given, the images are resized to the required image size by resizing with a bilinear interpolation.

\subsection{Data} \label{sec:data}
The authors of the original approaches use different content and style images, as well as different datasets for the training of the models in the model-based methods. In order to compare the replicated images, we use the images used in the original papers for the verification of the replication. 

Some of the images are available in the original GitHub repository. The images that have not been made accessible have been selected as a result of an image search. Links and license information are available in the implementation for these images. We have only used images whose license information allows them to be used in a scientific work. An overview of the sources for the content and style images used in this study can be found in the \appref{app:image_credits}. 

For the training of model-based methods, the different datasets used for the respective implementation are listed in \tabref{tab:datasets}. Each training image is preprocessed according to the steps specified in \paper{} or in \implementation{}. This includes for example image augmentation and cropping of the images. Information on this can be found in the tables in \appref{sec:parameters_appendix}.

The following datasets are required for this replication study:

\begin{itemize}
	\item The Microsoft Common Objects in Context (MS COCO) dataset contains 164K images. The size of the dataset is 25 GB \cite{LMB+2014}.\urlfootnote{Source}{https://cocodataset.org}{28.09.2021}
	\item The Large Scale Visual Recognition Challenge 2012 (ILSVR2012) is a subset of ImageNet containing 1000 categories and 1.2 million images. The size of the dataset is 154.6 GB \cite{RDS+2015}.\urlfootnote{Source}{https://image-net.org/challenges/LSVRC/2012/}{28.09.2021}
\end{itemize}

\begin{table*}[h]
	\caption{Overview of the datasets required for the replicated paper.}
	\label{tab:datasets}
	\centering
	\small
	\begin{tabular}{c|c}
		\hline
		\bfseries Methods & \bfseries Dataset\\
		\hline\hline
		\etal{Johnson} \cite{JAL2016} & MS COCO  \\
		\etal{Ulyanov} \cite{ULVL2016} & ILSVRC2012$^1$ \\
		\etal{Ulyanov} \cite{UVL2017} & ILSVRC2012 \\
		\hline
	\end{tabular}
\footnotesize{
	\\$^1$ Only the validation set is specified in \implementation{}.
}
\end{table*}

\subsection{Computional requirements}

All our final replication scripts were performed on a machine having a Titan RTX CUDA GPU with 24 GB memory. A GPU with a lot of memory is particularly needed for model-based approaches. Our CUDA version is $11.7$ and the Nvidia driver version is $515.43.04$. The implementation also requires different libraries, which are listed in the  \tabref{tab:libraries} with the version used. In addition, the datasets used must be available. 

\begin{table*}[h]
	\caption{Overview of the libraries used with the version used at the time of replication.}
	\label{tab:libraries}
	\centering
	\small
	\begin{tabular}{c|c}
		\hline
		\bfseries Library & \bfseries Version\\
		torch & 1.12.1\\
		torchvision &  0.13.1\\
		pystiche & 1.0.1\\
		pillow & 9.2.0\\
		numpy & 1.21.6\\
		more-itertools & 8.14.0\\
		\hline
	\end{tabular}
\end{table*}