\section{Introduction}
Humans have always been attracted and inspired by the art of painting. This attraction is explained by the interplay between the content and style of the painting, with which the artists create extraordinary visual experiences \cite{Glas2021}. Whether it is possible to teach this ability to a machine is still unknown, since solving this task with a traditional supervised learning approach is largely impractical. This is because the pairs of images needed to train a \gls{ML} model, the original image and an artistic representation of that image, rarely exist.
 
However, in recent years, there are artificial systems that generate artistic images of high perceptual quality based on \gls{DL} algorithms. This technique of recomposing images in the style of other images is called \gls{NST} and was introduced by \textsc{Gatys}, \textsc{Ecker}, and \textsc{Bethge} in $2016$ \cite{GEB2016}. An example of a content image receiving the style and a style image from which the style is transferred, as well as the result of the \gls{NST}, can be seen in \figref{fig:fig_nst}. 

The transformative power of this technique is that almost anyone can create and share an artistic masterpiece. This allows people all over the world to experiment with their own creativity \cite{Kel2018}. On 
\begin{figure*}[h]
	\centering
	\includestandalone{nst/example_images}
	\caption{Example of a Neural Style Transfer with the content image \contentimage{} (left), the used style image \styleimage{} (middle) and the stylised image \image{} (right).}
	\label{fig:fig_nst}
\end{figure*}
the other hand, the significance can also be observed in the commercial art world. In $2018$, Christie's featured Artificial Intelligence artwork sold at one of their auctions for $\$432.500$ \cite{2018}. Furthermore, the transfer of styles to recorded and live video opens many doors in the areas of design, content creation, and creative tool development. For example, \gls{NST} can be applied in various ways to photo and video editors, virtual reality, gaming, and commercial art \cite{Glas2021, Ioa2021}.

These multiple possible applications have led to \gls{NST} becoming a trending topic in academic literature in recent years. This is underlined by the number of citations to the initial paper \eqq{Image style transfer using convolutional neural networks} \cite{GEB2016}. In the past $6$ years, according to Google Scholar\urlfootnotename{Overview of the papers in which the paper or preprint is referenced}{https://scholar.google.de/scholar?oi=bibs&hl=en&cites=15430064963552939126,6343685530593283491,788840246532963346,18334534842043149041}{Google Scholar Link}{01.02.2022}, this paper has been cited over $5000$ times. 

The problem with such a rapid development of publications is the lack of standardisation, which is usually introduced after the initial boom. Without a standardisation, the approaches exist in different programming languages, as well as \glspl{DLF}. This makes a direct combination impossible and a comparison of the methods difficult, although the authors of the methods provide the source code of their approaches. 

During the review of the reference implementations, we also noticed that there are discrepancies between the published algorithm and the provided implementation. These range from incorrectly specified or not specified hyperparameters to minor implementation errors to significant changes in the algorithms compared to the publications. This makes an exact replication of the results more difficult or even impossible. We suggest that the discrepancies are not deliberate deceptions, but rather artefacts due to lack of standardisation. 

Based on these observations, we took two steps. First, we have introduced a standardisation for the implementation of \gls{NST}. For this purpose, the public \gls{NST} library with the name \texttt{pystiche} was created \cite{ML2020}. This library requires only a minimum of prior knowledge about \gls{NST} and \gls{DL} and is also flexible enough to combine the different approaches and does not limit the scope of action. The \texttt{pystiche} library is based on and is fully compatible with \texttt{PyTorch} \cite{PGM+2019}. 

In a second step, we replicated known \gls{NST} approaches using \texttt{pystiche}. In addition to \textsc{Gatys}'s, \textsc{Ecker}'s, and \textsc{Bethge}'s initialisation paper \cite{GEB2016}, we have replicated other image-based approaches in which a single image is stylised using an optimisation algorithm \cite{LW2016,GEB+2017}. Furthermore, we have replicated model-based \gls{NST} approaches \cite{JAL2016,ULVL2016,UVL2017}, where a model is learned that can subsequently be used for stylising any content images. These replicated approaches cover a wide range of current approaches and can be used as a basis for new approaches. In addition, the implementation of the replications facilitates a comparison of new approaches with the existing approaches. 

We have replicated a total of seven known \gls{NST} procedures based on the library \texttt{pystiche} in this replication study. Due to the fact that discrepancies between the paper and the reference implementation were found in each study, we replicated each approach in two different ways. One replication using only the information from the paper (\paper{}) and one replication using the default information from the reference implementation (\implementation{}). This highlights the open problem of lack of replicability and underlines once again the need for \gls{NST} libraries like \texttt{pystiche} beyond this study.

The paper is structured as follows. In the next section, the basic functionality of \gls{NST} is introduced. In the following third section, the implementations reproduced in this study and the reason for their selection are briefly presented. The fourth section explains the replication methodology and how we dealt with the discrepancies between the implementations and the paper. Finally, the results and their significance are discussed.