\usepackage{standalone}
\usepackage{lscape}
\usepackage{amsfonts}
\usepackage{xspace}
\usepackage{hyperref}
\usepackage{multirow}
\usepackage{hyperref}
\usepackage{adjustbox}

\usepackage{rotating,capt-of}

\graphicspath{{graphics/}}

% referencing commands
\newcommand*{\secname}{Section}
\newcommand*{\secref}[1]{\secname{}~\ref{#1}}

\newcommand*{\figname}{Figure}
\newcommand*{\figref}[1]{\figname{}~\ref{#1}}

\newcommand*{\tabname}{Table}
\newcommand*{\tabref}[1]{\tablename{}~\ref{#1}}

\newcommand*{\appname}{Appendix}
\newcommand*{\appref}[1]{\appname{}~\ref{#1}}

\newcommand{\urlfootnote}[3]{\footnote{#1, \url{#2}, retrieved on #3.}}
\newcommand{\urlfootnotename}[4]{\footnote{#1, \href{#2}{#3}, retrieved on #4.}}
% glossaries
\usepackage[acronyms]{glossaries}
\newcommand{\newacr}[4][]{\newacronym[
	sort={\ifthenelse{\isempty{#1}}{#2}{#1}},
	]{#2}{#3}{#4}}
\glsdisablehyper
\defglsdisplayfirst[\acronymtype]{\emph{\glsentrylong{\glslabel}} (\glsentryshort{\glslabel})}

\newacr{NST}{NST}{Neural Style Transfer}
\newacr{DLF}{DLF}{Deep Learning Framework}
\newacr{CV}{CV}{Computer Vision}
\newacr{GAN}{GAN}{Generative Adversarial Network}
\newacr{DL}{DL}{Deep Learning}
\newacr{ML}{ML}{Machine Learning}
\newacr{CNN}{CNN}{Convolutional Neural Network}
\newacr{MSE}{MSE}{mean squared error}
\newacr{SE}{SE}{squared error}
\newacr{MRF}{MRF}{\textsc{Markov} random field}
\newacr{VGG}{VGG}{Visual Geometry Group}
\newacr{GPU}{GPU}{Graphical Processing Unit}

% helpful commands
\newcommand*{\eqq}[1]{''#1''}

%math commands
\newcommand{\realnumbers}{\ensuremath{\mathbb{R}}}

\renewcommand{\matrix}[1]{\ensuremath{\boldsymbol{\mathrm{#1}}}}
\newcommand{\tensor}[1]{\ensuremath{\boldsymbol{\mathsf{#1}}}}

\newcommand{\transformer}{\ensuremath{\mathcal{T}}}

\newcommand{\paper}{\texttt{Pa\-per}}
\newcommand{\implementation}{\texttt{Im\-ple\-men\-ta\-ti\-on}}

\newcommand{\image}{\tensor{I}}
\newcommand{\contentimage}{\ensuremath{\image_\text{C}}}
\newcommand{\styleimage}{\ensuremath{\image_\text{S}}}
\newcommand{\loss}{\ensuremath{\mathcal{L}}}
\newcommand{\contentloss}{\ensuremath{\mathcal{L}_{\text{C}}}}
\newcommand{\styleloss}{\ensuremath{\mathcal{L}_{\text{S}}}}

\newcommand{\mean}{\ensuremath{\overline{\sum}}}
\newcommand{\transpose}[1]{\ensuremath{#1^T}}
\newcommand{\argmin}[2]{\ensuremath{\underset{#1}{\text{arg min}}\:#2}}
\newcommand{\argmax}[2]{\ensuremath{\underset{#1}{\text{arg max}}\:#2}}

\newcommand{\eqspace}{\,}
\newcommand{\Eqspace}{\quad}
\newcommand{\eqcommasep}{,\eqspace}

\newcommand{\eqtextdot}{\ensuremath{\text{.}}}
\newcommand{\eqtextcomma}{\ensuremath{\text{,}}}
\newcommand{\eqtextand}{\ensuremath{\text{and}}}
\newcommand{\eqtextwith}{\ensuremath{\text{with}}}

\DeclarePairedDelimiter{\@parentheses}{(}{)}
\newcommand{\parentheses}[1]{\ensuremath{\@parentheses*{#1}}}
\makeatletter
\newcommand{\etal}[1]{\textsc{#1} et al.\xspace{}}
\makeatother
\newcommand{\of}[1]{\parentheses{#1}}
\newcommand{\fun}[2]{\ensuremath{\text{#1}\of{#2}}}


\newcommand{\spatvec}[1]{\fun{spatvec}{#1}}
\newcommand{\spatvecinv}[1]{\fun{spatvec$^{-1}$}{#1}}
\newcommand{\gram}[1]{\fun{gram}{#1}}

\newcommand{\imagecredits}[3]{\bigskip#1, #2 \\\url{#3}}
\newcommand{\imagecreditshref}[4]{\bigskip#1, #2 \\\href{#3}{#4}}


% footnote referenz multiple times
\newcommand{\footlabel}[2]{%
	\addtocounter{footnote}{1}%
	\footnotetext[\thefootnote]{%
		\addtocounter{footnote}{-1}%
		\refstepcounter{footnote}\label{#1}%
		#2%
	}%
	$^{\ref{#1}}$%
}

\newcommand{\footnoteref}[1]{%
	$^{\ref{#1}}$%
}

\usepackage{pifont}
\newcommand{\cmark}{\ding{51}}%
\newcommand{\tcmark}{(\ding{51})}%
\newcommand{\xmark}{\ding{55}}%
