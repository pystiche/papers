\documentclass[class=journal]{standalone}

\usepackage{physics}
\usepackage{amsmath}
\usepackage{tikz}
\usepackage{mathdots}
\usepackage{yhmath}
\usepackage{cancel}
\usepackage{color}
\usepackage{siunitx}
\usepackage{array}
\usepackage{multirow}
\usepackage{amssymb}
\usepackage{gensymb}
\usepackage{tabularx}
\usepackage{booktabs}
\usetikzlibrary{fadings}
\usetikzlibrary{patterns}
\usetikzlibrary{shadows.blur}
\usetikzlibrary{shapes}

 
\begin{document}
\tikzset{every picture/.style={line width=0.75pt}} %set default line width to 0.75pt   
\begin{tikzpicture}[x=0.7pt,y=0.7pt,yscale=-1,xscale=1]
   \draw (288.25,67.5) node  {\includegraphics[width=129.38pt,height=89.25pt]{images/style1.jpg}};
   \draw (115.75,67.5) node  {\includegraphics[width=129.38pt,height=89.25pt]{images/content.jpg}};
   \draw (460.75,67.5) node  {\includegraphics[width=129.38pt,height=89.25pt]{images/result.jpg}};
   \draw [line width=2.25]    (193,69.2) -- (223, 69.2) ;
   \draw [line width=2.25]    (208,84.2) -- (208,54.2) ;
   \draw [line width=3]    (355,75.7) -- (383,75.7) ;
   \draw [line width=3]    (355,62.7) -- (383,62.7) ;
\end{tikzpicture}
\end{document}